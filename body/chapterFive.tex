% !Mode:: "TeX:UTF-8"

\section{研究方案及进度安排,预期达到的目标和取得的研究成果}
\subsection{研究方案}
研究目标为设计一种适用与混合载波通信系统的同步方案,包括最佳采样时刻同步、时域同步和载波频率同步。比较的标准为同步概率和同步参数估计的均方误差(MSE)。同步方式计划采用基于辅助序列的同步方法,首先设计同步序列结构,比如采用~CAZAC~序列及其共轭,然后设计接收端定时度量函数,使得定时函数在同步位置保持尖锐,即得到较好同步效果的,然后根据序列结构设计频偏度量函数,估计出时域偏差与频率偏差,完成同步过程。

研究过程大致分为三步。

第一步大量查阅文献,进行同步算法的总结综述,对不同算法进行理论分析和比较,明确适用范围以及优缺点。

第二步探索现有同步算法与混合载波通信系统的兼容性,选取适用于混合载波通信系统的同步算法,并尝试应用在混合载波通信系统中,通过仿真来进行性能分析,比较各实现方法的计算复杂度、同步速度、同步参数估计的均方误差。

第三步设计适用于混合载波通信系统的新型同步序列结构,设计在同步时刻更加尖锐的定时度量函数,给出混合载波通信系统的同步方案。

\subsection{预期达到的目标和取得的研究成果}
完成混合载波通信系统同步方案的设计与性能仿真,从而更加完善基于~WFRFT~混合载波通信系统的研究。

\subsection{进度安排}
\begin{table}[htbp]
\centering
\caption{进度安排}\label{table3}\vspace{-0.5em}\zihao{5}
\begin{tabularx}{0.8\textwidth}{lX}
\toprule
Name & Alternate names, and names for sets of the given size\\
2017.6~2017.8 & 查阅先关文献,确定课题内容及方案。\\
2017.8-2017.10 & 完成5.1节中第一步,对现有同步算法进行综述与分析。\\
2017.10-2017.12	& 完成5.1节中第二步,探索分析现有同步算法,以及应用在混合载波通信系统的可能性,并将适用的算法尝试应用在混合载波通信系统中。\\
2017.12-2018.2 & 继续5.1节中第二步,对应用于混合载波通信系统的同步算法进行进行仿真对比分析。\\
2018.2-2018.5 & 完成5.1节中第三步,设计混合载波通信系统同步解决方案,并进行可行性分析与性能仿真。\\
2018.5-2018.7 & 撰写毕业论文,准备答辩。\\
\bottomrule
\end{tabularx}
\end{table}


\section{为完成课题已具备和所需的条件和经费}
近年来关于混合载波通信系统的研究比较多,可提供研究思路的参考的文献很容易获取。基本理论比较成熟也为本项研究奠定了一定的理论基础。本课题来源于国家重点基础研究发展计划(973计划)——异构网络协同信号处理理论与方法,项目正顺利进行,且已经有了一些成果与经验,也会为本课题研究的开展提供有力的支撑。为完成本课题,还需要查阅大量文献资料,并且可能需要借阅或者购买一些参考书籍。本课题所需的研究经费和实验设备条件已具备。


\section{预计研究过程中可能遇到的困难和问题,以及解决的措施}
本课题中基于混合载波通信系统的同步方案设计还是比较新颖的,所以研究中同步偏差对通信系统的影响,以及各种同步算法的原理与同步定时函数的设计会使用较多的数学方面的知识。因而数学理论的学习以及公式的推导可能需要大量的时间。在研究的过程中,需要查阅相关文献,并及时与老师及实验室师兄、师姐们沟通。
