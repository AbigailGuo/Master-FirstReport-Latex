% !Mode:: "TeX:UTF-8"

\section{课题来源及研究的背景意义}
\subsection{课题的来源}

本课题来源于国家重点基础研究发展计划(973计划)——异构网络协同信号处理理论与方法(课题编号2013CB329003)。针对频谱资源的紧张与日益增长的对高速率、高质量通信需求之间的矛盾,文献\cite{mei2010}中提出一种混合载波通信系统解决方案,它是单载波与多载波体制相互融合的一种混合载波体制,在保密性、抗干扰性上有较好表现,文献\cite{hui2015_177_179}表明在接近真实复杂信道环境的双选信道模型下,在最佳变换阶数传输时,其性能优于其他两种载波体质,这种载波体质的融合正是当前通信领域的一个研究热点与趋势。目前,在加权类分数傅里叶的研究中,以加权分数傅里叶变换(Weighted-type fractional Fourier transform,
WFRFT)技术为核心的混合载波通信系统虽然还没有得到广泛应用,却是当前通信领域的研究热点之一。


\subsection{课题研究背景及意义}

自从法国科学家~J.Fourier~在1807年为求解热传导方程而首次提出著名的傅里叶分析以来,经过一百多年的发展,傅里叶变换在科学研究与工程应用中都起着重要的作用,是一种基础有效的分析工具。在通信系统中,信号的时间和频率是相互关联的,在时域和频域内都包含着信号的全部信息,只是观察和分析的角度不同,一个信号通过傅里叶变换可以从时域转换到频域,从而在频域中观察、分析和处理信号。这种分析问题的思想具有里程碑的意义。然而随着理论研究与工程应用的不断发展,在特定的应用中,不断暴露出傅里叶变换本身的局限性。傅里叶变换的基函数是一组指数函数,由于其是时间的一次函数,傅里叶变换比较适合分析变化平稳的信号。然而生活中存在的大多数信号都属于非平稳信号,若用傅里叶变换分析,仅仅是将信号从整个时域变换到频域,这对于分析信号在时域的局部变化特性有较大局限性。反之傅里叶逆变换也无法有效分析信号在频域的局部特性。这种条件下单纯的从一种域中分析信号显然效果不佳,对于这种“时频联合”的问题理应用一种“时频联合”的手段来进行分析。作为傅里叶变换的广义形式,分数傅里叶变换(Fractional Fourier Transform,FRFT)的出现大大弥补了傅里叶变换的不足,它不但继承了传统傅里叶变换的基本性质,且具有后者不具有的优良特性。分数傅里叶变换后的信号能够同时包含时域和频域信息,能够在介于时域和频域之间的分数域上分析信号,能够显示出信号从时域逐渐变化到频域的所有特征,因此是一种很好的时频分析工具。这种新思想、新工具采用时频联合的方式来对通信过程中的波形设计、时频分析进行描述,其中加权的分数傅里叶变换是分数傅里叶变换的变化形式,它是一种特殊的混合载波技术,它的时域表达形式对应传统通信系统中的单载波模型,频域表达形式对应通信系统中的多载波模型。这种把不同载波形式和通信体制混合在一起的方式可以避免单独追求某一方面的好处带来的弊端,起到折中的效果。这是从有效性出发的角度考虑,另一方面,从可靠性的角度分析,由于经过~WFRFT~处理后的信号在星座图、时频平面上都具有独特的性质,而此性质可以用来帮助对抗通信过程中非目的接收机对正常用户通信的干扰,从而在抗截获和抗干扰方面有较好的性能。

在实际移动通信当中由于更高的载波频率、发射机与接收机之间更快的相对移动、发射机接收机的不稳定性(如晶振的漂移导致的采样抖动)等,多普勒频移效应更明显,信道变化更快,信号是一种频率随时间快速变化的非平稳信号,适合利用分数傅里叶变换来分析。因而将分数傅里叶变换应用在通信系统中,利用其具有的诸多傅里叶变换不具有的优良特性,可以克服傅里叶变换在现有通信系统中的一些应用局限,具有非常广阔的应用前景。文献中\cite{mei2017_38_44}已经提出将加权类分数傅里叶变换用作通信系统过程中,并且给出了基于~WFRFT~的混合载波通信系统框架和物理实现过程,分析了单参数和多参数分数傅里叶变换通信系统的抗干扰抗截获性能,以及与单载波多载波通信系统在衰落信道下的比较。

~WFRFT~作为一种可以对抗通信过程中的干扰和截获的技术,已经在通信领域中呈现出了较好的研究价值和应用前景。然而一个通信系统想要正常工作,达到期望的性能指标,完成较好的同步过程是必要的,而目前针对混合载波通信系统的同步技术尚处于较为空白的研究阶段。本文将主要研究基于混合载波通信系统的同步技术。

