% !Mode:: "TeX:UTF-8" 

\section{模板的其它说明}

\subsection{单层罗列环境}
哈工大学位论文一般可采用两种罗列环境:一种是并列条目有同样标签的~\verb|itemize|~罗列环境,另一种是具有自动排序编号符号的~\verb|enumerate|~罗列环境。这两种罗列环境的样式参数可参考图~\ref{list}。
\begin{figure}[htbp]
\centering
\includegraphics[width = 0.6\textwidth]{list}
\caption{罗列环境参数示意图}\label{list}\vspace{-1em}
\end{figure}
通过调用~enumitem~宏包可以很方便地控制罗列环境的布局,其~format.tex~文件中的~\verb|\setitemize|~和~\verb|\setenumerate|~命令分别用来设置~\verb|itemize|~和~\verb|enumerate|~环境的样式参数。采用~\verb|itemize|~单层罗列环境的排版形式如下:
\begin{itemize}
\item 第一个条目文本内容
\item 第二个条目文本内容
\item 第三个条目文本内容
\end{itemize}
其代码如下
\begin{verbatim}
\begin{itemize}
  \item 第一个条目文本内容
  \item 第二个条目文本内容
  ...
  \item 第三个条目文本内容
\end{itemize}
\end{verbatim}
采用~\verb|enumerate|~单层罗列环境的排版形式如下:
\begin{enumerate}
\item 第一个条目文本内容
\item 第二个条目文本内容
\item 第三个条目文本内容
\end{enumerate}
其代码如下
\begin{verbatim}
\begin{enumerate}
  \item 第一个条目文本内容
  \item 第二个条目文本内容
  ...
  \item 第三个条目文本内容
\end{enumerate}
\end{verbatim}

\subsection{定理定义}

若需要书写定理定义等内容,而且带有顺序编号,需要采用如下环境。除了~\verb|proof|~环境之外,其余~9~个环境都可以有一个可选参数作为附加标题。

\begin{center}\vspace{0.5em}\noindent\zihao{5}\begin{tabularx}{0.7\textwidth}{lX|lX}
定理 & \verb|theorem|~环境 & 定义 & \verb|definition|~环境 \\
例 & \verb|example|~环境 & 算法 & \verb|algo|~环境 \\
公理 & \verb|axiom|~环境 & 命题 & \verb|proposition|~环境 \\
引理 & \verb|lemma|~环境 & 推论 & \verb|corollary|~环境 \\
注解 & \verb|remark|~环境 & 证明 & \verb|proof|~环境 \\
\end{tabularx}\end{center}