% !Mode:: "TeX:UTF-8"

\section{国内外在该方向的研究现状及分析}
\subsection{加权类分数傅里叶变换研究现状}
20~世纪以来,尤其随着离散傅里叶变换算法的提出,傅里叶变换广泛应用在各种技术领域中。但是,在某些特定的应用中,傅里叶变换也逐渐表现出一定的局限性,特别是涉及到非平稳信号的分析与处理方面。因此,短时傅里叶变换、小波变换以及分数傅里叶变换等信号处理手段相继出现,用以在处理信号的过程中保留并体现出非平稳信号的特性。

FRFT~的研究可以追溯到~1929~年,N.Wiener~率先将傅里叶变换的特征值进行分数化表示。随后~1937~年~E.U.Condon~的研究给出了分数傅里叶变换的基本定义。到了~1961~年,V.Bargmann~在~E.U.Condon~的研究结果基础上,尝试将~FRFT~用于研究解析函数的~Hilbert~空间;在~1980~年,V.Namias~从特征值与特征函数的角度,重新对~FRFT~进行了新的定义;A.C.Mcbride~随后利用积分对~V.Namias~定义的~FRFT~进行了研究分析,奠定了后续研究的理论基础,我们将此类分数傅里叶变换称为~Chirp~类分数傅里叶变换(Chirp-type fractional Fourier transform, CFRFT),即经典分数傅里叶变换。

在此之后,研究者们在这类分数傅里叶变换理论方面做出了很多贡献,逐步完善了分数傅里叶的基本定义及性质,分析比较这种变换与其它时频分析工具之间的联系,研究其数值计算与快速算法。特别是在处理~Chirp~类非平稳信号时表现出的处理能力,受到国内外研究员的广泛关注。

直到~1995~年,加权类分数傅里叶变换(Weighted-type fractional Fourier transform, WFRFT)被~C.C.Shih~提出,这是一类新的分数傅里叶变换,与经典~FRFT~不同,它是由态函数加权的形式构成。特殊的当态函数个数取~4~时,称之为四项加权分数傅里叶变换。而又按照周期及加权项数目的不同,存在多种形式定义的加权分数傅里叶变换。如~S.T.Liu~在~1997~年提出加权变换系数的周期可以扩展为~4~的整数倍;B.H.Zhu~在~2000~年又提出将加权系数周期扩展到大于~4~的任意整数;直到~2005~年,这些不同定义形式的分数傅里叶变换才被~Q.W.Ran~统一起来,都可以归类为一种广义多参数分数傅里叶变换。于是随着~WFRFT~加权系数的推广,理论逐渐完整。目前,研究经典分数傅里叶变换学者很多,成果也比较明显,在通信领域有一定的应用。


\subsection{加权类分数傅里叶变换在通信领域的应用}
在国外对~WFRFT~基础理论研究的基础上,国内学者重点致力于研究~WFRFT~在通信系统中的应用,也就是所谓的混合载波通信系统。这方面的研究已经取得了一些列的成果。由于~WFRFT~变换性质的特殊性,将其应用在通信系统中,其对信号的处理过程以及处理后的时域与频域信号表达形式与现有通信系统中的单载波与多载波调制过程相一致,因此基于~WFRFT~的通信系统可以视为是单载波与多载波系统融合的混合载波通信系统,而且由于调制阶数的周期连续性,可以通过调节调制阶数直接实现单载波多载波之间通信模式的切换,而不需要重新设计硬件。而这中混合载波通信体制,可以通过设计合适阶数等手段来适应信道,抵抗干扰,进而继承两种载波体制的优势,逼近信道容量[2]。另一方面,有学者的研究表明,可以设计一种多参数且参数可时变的~WFRFT~通信系统,当非目的接收端未知传输参数时,无法解调信息,实现保密传输的能力。

文献[14]基于~WFRFT~提出了一种新的信号收集与恢复的方法,WFRFT~作为香农采样定理以及傅里叶变换的扩展,能够对非带限进行完全恢复,文中同时给出了一组信号分解的正交基。文献[15]率先将~WFRFT~应用在了通信系统当中,并且建立了基于~WFRFT~的一种通信系统模型,这种通信系统同时包含有单载波与多载波方式,通过仿真得出这种混合载波通信系统相对于单载波和多载波通信系统具有更好的性能。文献[16]利用~WFRFT~对信号进行变换以隐藏信号星座,进而对通信系统进行加密。传统的加密通常在信源编码处实现,但是为了隐藏数据的星座将会增加噪声,通过这种方法可以在不引入噪声的前提下,增强系统容量和安全性。文献[17]指出~WFRFT~通信系统不仅同时包含有~SC~和~MC~信号,该系统作为一种统一方案,继承了两种载波体制的性能,这其中包括了高峰均比(PARP)问题,文中指出~WFRFT~系统的~PARP~处于单载波系统和多载波系统之间。并指出可以通过调节~WFRFT~的调制参数来控制~PARP,同时现有的方法也可以引入到~WFRFT~通信系统中用以降低~PARP~。文献[18]提出了一种基于高阶累积量混合载波模式识别方案。文献[19]基于~WFRFT~对通信信号进行波形设计,以抑制时频双选信道,同时指出经过波形设计的通信系统在接收端不适用均衡也能获得很好的性能。文献[20]介绍了一种基于双层加权类分数傅里叶变换(DL-WFRFT)的新型抗截获通信系统。与传统的物理层抗截获技术,如时间跳频或频率跳频,~DL-WFRFT~在保证安全前提下具有较高的频谱效率。同时由于采用特殊的双层转换结构,比单层变换系统对调制参数更为敏感。因此,该安全系统的抗截获性能更好。文献[21]提出了一种抑制高多普勒频移时频双选信道干扰的方案,该方案提出了一种混合载波体制下基于最小均方误差的迭代均衡算法,通过仿真得出,该方案的性能能够由于采用该算法的单载波以及多载波通信系统。文献[22]将混合载波体制与~CDMA~技术进行组合。这种系统可以实现单载波与多载波之间的过渡。此外,提出的~HC-CDMA~技术在频率选择性衰落信道和单频干扰信道中优于~SC-CDMA~和~MC-CDMA~系统。文献[23]中,将加权分数傅里叶变换技术作为一种预编码方案,应用到基于~DFT~的通信系统中以抑制窄带干扰。WFRFT~可以通过控制调制参数退化为单载波(SC)或多载波(MC)频分多址传输系统,这也能够作为一种在频域规避窄带干扰的策略,通过对~WFRFT~的参数控制,可以得到优于~SC~和~MC~的抗窄带干扰性能。文献[24]中指出,基于分数傅里叶变换的混合载波体系和基于能量传播的调制变换(EST)已成为抑制时间和频率双选信道引起的载波间干扰的有效解决方案。文中提出了一种迭代的频域最小均方误差均衡算法,将其应用在基于~WFRFT~和~EST~预编码的系统中,可以对干扰抑制效果和复杂度进行折中,仿真结果表明该算法在误码率方面由于现有的迭代~MMSE~均衡算法文献[25]提出将以~WFRFT~为基础的混合载波通信系统应用在水声信道中作为调制解调方案,由于传统的~OFDM~系统容易受到子载波间干扰,而对于混合载波系统,通过部分~FFT~解调和加权补偿可以很好的抑制子载波间干扰,通过仿真可以看出在较小的复杂度下可以取得较好的干扰抑制效果。文献[27]中,提出了一种并行组合的扩频加权分数傅里叶系统对物理层进行加密。在这个系统中,发送数据被分为两组。首先基于第一组数据产生一组伪随机序列的,然后用来传播和加密第二组数据。文中指出该系统是通过信息本身进行传播和动态加密,因此可以获得较高的带宽效率和较高安全性能。文献[28]设计了一种基于时频双选信道的最佳~WFRFT~调制参数的选择算法,通过侦测信道状态,可以选取最佳调制参数进行通信,能够匹配信道,使干扰平均化,进而达到最优的误码率性能。

\subsection{通信系统同步技术理论与研究现状}
众所周知,同步过程对于任何一个通信系统来说都是至关重要的。通信系统中最重要的两个问题一个是有效性问题,一个是可靠性问题,而保证有效性和可靠性前提是要求通信系统正常工作,正确的同步是通信双方正常通信的前提条件。近些年国内外在同步方面研究的重点主要是集中在~OFDM~系统当中,由于基于~WFRFT~技术形成的混合载波通信系统同时融合了单载波与多载波通信体制,因此这些研究中有一些也适用于混合载波通信系统系统。

根据不同的分类、目的和用途,同步可以从不同角度来区分。按照同步实现的功能区分,同步可以分为载波同步、帧同步、位同步等。按照同步实现的方式来划分,可以分为依靠辅助前导序列实现的同步,和不依靠辅助序列实现的盲同步。不依靠辅助序列的同步方法主要利用发送信号自身的结构特点来进行同步,优点是实现成本低,不占用额外频谱资源,但由于信号在在进行传输过程中会受到噪声等各种干扰,信号的结构会受到影响,故同步精度会有所下降。基于辅助序列的同步算法基本思想是在信号中插入收发双方均已知的训练信号,接收端通过观测窗口进行滑动捕获,来寻找正确定时位置。此种方法计算复杂度低,捕获速度快,有较好的同步性能,但同时由于在信号中插入了额外的同步序列,会增加系统开销。

一般情况下,为了得到更好的性能,接受端都采用相干解调的方式,即在接收端产生一个跟发射端的载波同频同相的载波,此载波称为相干载波,在单载波系统中普遍通过锁相环实现,如科斯塔斯环、四项松尾环等,而在多载波通信系统中,则是先将信号进行频谱搬移,在基带对信号的频偏相偏进行估计后再对信号补偿。位同步是在载波同步进行的同时另一个需要考虑的问题,信号在通信过程中信号会经过复杂的信道环境,并且在接收端会经过的一系列滤波处理后,此时信号在一个码元周期内不再为恒定值,此时经过再进行模数变换,需要在最佳抽样判决时刻对接收信号进行采样,才能保证在后续的解调过程正常进行并使系统达到最优的性能。此过程成为位同步或者码元同步。正确的位同步是正确的抽样判决的先决条件。不论通信体制,都是将不同数目的码元封装在一起,组成一个帧来作为传输的单位,所以在接收信号的时候,也需要知道这些帧的起止时刻。在~OFDM~系统中,对于同步算法主要分为两大类,一类是基于循环前缀的定时同步算法,另一类是基于训练序列的算法。文献中给出了基于最大似然准则(ML)的同步二维判决函数,主要是利用循环前缀在时域上的重复特性,依据自相关,在二维判决域中同时确定信号的定时位置与频偏。文献中对其进行了改进,给出了一种简化的一维函数判决方式,计算量大大降低,性能相似。而基于训练序列的经典定时同步算法主要包括~SC~算法、Minn~算法、Park~算法等等。文献在~Minn~算法的基础上提出了利用~PN~序列加权训练序列的同步算法,得到的同步相关函数峰值更尖锐,同步精度更高。文献中提出基于共轭序列的同步方法,抗噪声性能更好。

